

\section{ Problem Statement}

Since airline seats are perishable and consumer behavior varies greatly, effective ticket pricing remains a significant challenge in the highly competitive and dynamic airline industry. The severe constraints that airlines must deal with include limited seat inventory, non-storable service offerings, and fluctuating demand over time. India's domestic airline market largely depends on dynamic pricing because of different customer segments, rising consumer awareness, and government laws requiring fare transparency. 

\vspace{1em}

As a monopolistic seller, the airline offers a single fare class for a fixed number of seats in a finite-horizon dynamic pricing problem. By selecting from a limited range of predetermined price points, ticket prices are dynamically set over the selling horizon to maximize revenue. At any given moment, the price is the same for every customer. There is no penalty for selling out early, and the airline does not restock seats. This arrangement mimics a real-world situation where demand for a given flight's tickets comes from both myopic (impulsive) and strategic (patient) buyers during a limited time frame. 

\vspace{1em}

Myopic consumers are more likely to buy immediately if the price matches their valuation, whereas strategic consumers may postpone their purchase in anticipation of a price reduction. They frequently fail in such environments because traditional pricing strategies rely on fixed pricing rules and static demand estimates that cannot adjust to quickly changing conditions. Furthermore, previous research has generally assumed that demand distributions and consumer behavior are known ahead of time, which is rarely the practice case. These restrictions are made even more noticeable in the Indian aviation market because of the impact of regulatory actions, consumer price sensitivity, and significant demand fluctuations during emergency or seasonal times. 

 \newpage
\section{Objectives}\label{obj}
\textbf{Understanding Dynamic Pricing Mechanism:}  
To check the composition and operation of the airlines' current dynamic pricing systems. By understanding the theoretical foundations of dynamic pricing, the algorithmic techniques that support real-time price adjustments and their applications in the aviation sector can be delivered. How Indian airlines adjust these strategies in response to market-specific elements like regulatory oversight and competitive pressure will receive special attention. 

\vspace{1em}

\textbf{Check Indian Market Fit:}  
To evaluate the suitability of current strategies, particularly those created in Western contexts, for the domestic airline sector in India. It entails investigating limitations such as price caps, governmental regulations, market fit and regional variations in consumer behavior. The objective is to ascertain whether localized adjustments are required for efficacy or whether the current dynamic pricing models can be successfully adapted to Indian conditions. 

