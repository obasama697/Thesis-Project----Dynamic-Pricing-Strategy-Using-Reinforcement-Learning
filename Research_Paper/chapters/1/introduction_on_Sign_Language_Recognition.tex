%\chapter{Introduction}
%\vspace{-1cm}
%\noindent \rule{6.6in}{0.01in}


\section{Introduction}\label{Chapter_1_Motivation}

The more the business environment gets competitive, dynamic pricing strategies settles itself in as it has emerged as a sophisticated tool for improving and maximizing on the revenues while serving the consumers optimally. Such strategies whereby the prices are continually changed based on factors like market demand and competition is one of the most common ones which is currently being practiced is the dynamic pricing strategy. A good example is the airline industry where items on offer and demand have constant elasticity and inventory is perishable. Since customers are not fully rational in their buying behavior, airlines have to make pricing decisions that not only would generate maximum revenues from ticket sales but also take full account of strategic customers who are likely to postpone their bookings until the right price is offered and myopic customers who are likely to book tickets as soon as the prices are reflected on the websites.

\vspace{1em}

Over the past few years, the players in the airline industry of India have felt the effects of growth mainly attributable to factors such as expansion in the economy, governmental support, and increase in the class of people considered as middle-class earners. But the corporate world has witnessed a relationship between the price and factors like market demand, seasons, operating cost especially the fuel price, consumer behavior, etc. in the airline's ticket price in India. Although prior literature has considered the topic of dynamic pricing in different contexts, this research will seek to extend its understanding, more potently, to the Indian context, exploring the factors that influence ticket prices in airlines, dynamic pricing methodologies employed by these airlines and the ways in which the behavior of consumers impacts purchasing decisions.

\vspace{1em}

The airline industry in India has grown to unprecedented levels over the last ten years. Currently India is the globe’s third largest domestic aviation market, due to growth in disposable income, increase in middle income earners and demand for air transportation. IndiGo, Air India, SpiceJet, Vistara, and GoFirst have tailored their fleets and network to better serve a steadily increasing domestic demand. Government’s UDAN scheme targeting the ‘any Indian citizen’ to fly more often by easing access to aviation has further propped up regional connectivity offered by waiving subsidies to the airlines for the regions.

\vspace{1em}

However, the future of the airline companies in India appears to be threatened by various uncertainties; fluctuations in fuel prices, dominant competition and unsteady passenger traffic among them. Dynamic pricing comes in handy for these challenges since it allows the airlines to adapt fares on living factors such as demand, booking, and competitor’s fares. Nevertheless, the Indian market is prospective but it demands particular attention to price sensitivity, regulation issues, and its buyers who can be both visionaries and impatient passengers.

\vspace{1em}

Consumer behavior is therefore crucial when the airlines are setting their pricing strategies into place. Due to the wide spectrum of price sensitivity among the consumers in India, the strategic balance to be kept by the airline players is relatively higher because on the one hand, they have to lure the early-booking strategic customers and on the other hand, the myopic customers. Strategic consumers, who keep waiting for some discount, introduce confusion into the demand uncertainty making the pricing process complicated. Inexperienced travelers, on the other hand, are likely to book with little consideration for the overall potential of the prices to change as the flight approaches, meaning that it is critical for airline companies to use the right fare timing strategies to their advantage.

\vspace{1em}

To a certain level, this paper is unique and it has enhanced the understanding of dynamic pricing in the Indian airline industry regarding the nature of strategic and myopic customer behavior, the use of machine learning models and artificial intelligence in revenue management. We will also be focusing on the various measures undertaken by the Indian government for the growth of the aviation industry with regard to pricing and consumers. Through the identification of main factors that influence air ticket pricing and analysis of dynamic pricing strategies, the study will be useful to practitioners and scholars interested in the pricing model improvement in the context of the Indian airline business.





\section{Key Concepts}\label{Chapter-1-Deep Reinforcement Learning in Cooperative Spectrum Sensing} 
In a highly competitive aviation industry, dynamic pricing has emerged as a vital tool for maximizing revenue and managing perishable inventory like airline seats. Indian airlines increasingly rely on data-driven strategies to respond to fluctuating demand, customer behavior, and market competition. This study focuses on the Indian domestic airline sector, exploring how strategic and myopic consumer behaviors affect pricing decisions. By leveraging reinforcement learning and machine learning models, the research aims to enhance fare optimization while addressing regulatory guidelines and consumer fairness. It also considers the government's role in shaping the industry through initiatives like UDAN and pricing transparency regulations. Before proceeding to unravel the dynamics of dynamic pricing strategies and the consumers’ buying behavior in the airline industry, there is the need to provide the meanings and understandings of some concepts that are of paramount importance to this study. 

\vspace{1em}
\textbf{Dynamic Pricing in Airline Industry}: Dynamic pricing is the strategy businesses use to set the price of products or services on different factors such as demand, competition, market condition, and time of purchase. This approach proves helpful in revenue maximization since companies can charge their customers the highest prices during demand peaks and offer the lowest prices during demand troughs. For the airline industry, the issue of dynamic pricing is especially sensitive, as that industry relies solely on seat sales as its product is non-storable  once the plane takes off, the unsold tickets are of no use. Airlines use dynamic fares to satisfy demand, optimize fares, and charge idle fares without considering daily and weekly market trends. 

\vspace{1em}
\textbf{Strategic vs. Mayopic Consumers}: Airlines usually encounter two unique groups of consumers: strategic and myopic consumers. Budget passengers are those people who know about the changes in prices and can afford to balance the current and future prices to buy commodities. They may postpone their purchase if they think prices will reduce nearer the traveling time. These passengers cause demand uncertainty mainly because their purchasing behavior depends on expected price levels, market saturation, and seasonal factors. However, myopic passengers make purchase decisions now and at that ticket price. They are not so bothered about future price drops and are in a position to purchase as long as the price matches the price they are willing to pay.

\vspace{1em}
\textbf{Role of AI and Machine Learning}: AI is defined as the building of complex systems to increase the capability of an entity to carry out operations that previously needed intelligence, which is the mental processes involved in learning, problem-solving solving, or decision-making. Artificial Intelligence (AI) includes an ML, which creates an algorithm that takes the system to learn and enhance the performance by its own experience of the data fed into it rather than being programmed. AI and ML are applied in the airline industry to determine customer behavior, suggest the correct fares, and improve revenue-generating processes. These technologies help airlines make large data sets for faster processing for operational and tactical purposes and to vary fare strategies depending on target markets’ conditions and trends.

\vspace{1em}
\textbf{Application of MDP for Fare Optimization}: A Markov Decision Process (MDP) is a model of decision-making that incorporates a stochastic element but where the decision maker chooses actions that determine the transition probabilities. Stochastic programming is generally used in operations research and artificial intelligence to assess pricing strategies under uncertainty, including MDPs. In the context of airline ticket pricing, MDPs can guide airlines to which action would help increase ticket price or reduce it based on current demand, inventory conditions, and customer behavior, thus weighing short-term revenues against long-term profitability.


